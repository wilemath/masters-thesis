\documentclass[../main.tex]{subfiles}

\title{Error Correction}
\author{Jack Wile}
\date{}

\begin{document}

Here we discuss correctability and privacy for quantum channels, now taken
to generality of $Op^*$-algebras as defined in the preliminary section. We
are interested in two kinds of each notion; the exact case and the 
approximate case. The former is merely an algebraic notion, and is little 
different than in the bounded case, save for the generality of the objects.
As discussed in the previous section however, we have a variety of 
generalizations of the operator norm which may or may not agree in the 
unbounded case, which leaves how best to think of correctability 
potentially more ambiguous, certainly compared to the setting which we are
trying to generalize.

It's worth noting still however that even in the bounded case there remains
some ambiguity with what the best notion of correctability and privacy are. First, let's state the definitions we are aiming to extend to unbounded 
algebras.

\begin{definition}

	Let $\mc{E}: \msa \to \msb$ be a quantum channel of \cstar-algebras. A
	\cstar-subalgebra $\ms{N} \subseteq \msb$ is said to be $\textbf{private}$
	for $\mc{E}$ if \[\mc{E}(\msa) \subseteq \ms{N}'.\] Similarly, we saw that 
	$\ms{N}$ is $\varepsilon$-private for $\mc{E}$ if there exists a 
	quantum channel $\mc{F}: \msa \to \msb$, private for $\ms{N}$, such that 
	\[\norm{\mc{E} - \mc{F}}_{\text{cb}} < \varepsilon.\]

\end{definition}

\begin{definition}

	Let $\mc{E}: \msa \to \msb$ be a quantum channels of \cstar-algebras. 
	a \cstar-subalgebra $\ms{N} \subseteq \msb$ is $\textbf{correctable}$
	for $\mc{E}$ if there exists a quantum channel $\mc{R}: \ms{N} \to \ms{A}$ 
	such that \[\mc{E}\mc{R} = \mc{I}_\ms{N}.\] $\ms{N}$ is $\varepsilon$-correctable
	for $\mc{E}$ if instead we have 
	\[\norm{\mc{E}\mc{R} - \mc{I}_\ms{N}}_{\text{cb}} < \varepsilon.\]

\end{definition}

Though we cannot write down the completely bounded norm for an unbounded channel, we can
topologize the channel in a manner which is both more general and captures the 
approximation up to "arbitrary amplification" in a manner similar to the completely 
bounded norm.

However, completely bounded convergence or pertubation is arguably too strong of a 
requirement to describe certain physical pertubations. It is then perhaps desirable to 
develop a correspondence between privacy and correctability in a weaker setting.

We also note that in the above definitions are often restricted to the case where 
$\msa$ is a von Neumann algebra, but the definition makes sense at this level of 
generality and is immediately analogous to our $Op^*$-algebra setting.

\subsection{$\msa$-amplifications}

We aim to work with correctability and privacy in the sense of one of a few possible 
locally convex topologies. It makes sense then to describe these in terms of the 
seminorms of whatever topology we choose. What isn't immediately clear is how to capture
the 'amplified' nature of the completely bounded topology, and so first a few things are
in order.

\begin{definition}

Let $\msa$ be an $Op^*$-algebra. We denote by $\msa_\infty = \msa \otimes M_\infty$ 
the space of infinite $\msa$-matrices, called the $\msa$-amplifications.

\end{definition}

It is an algebraic matter how one makes a $\ast$-algebra out of this space, but
what space should this act on? Let $\ms{D}_\infty$ be the space of 
finitely supported, $\ms{D}$-valued sequences. This is a pre-Hilbert space under the 
usual 'direct sum' inner product for Hilbert spaces. $\msa_\infty$ acts on 
these sequences in the usual manner. 
It is clear that $\msa_\infty$ is an $Op^*$-algebra on $\ms{D}_\infty$.

What is advantageous with this choice of domain is the stability of completeness of the 
domain with respect to the topologies induced by $\msa$ and $\msa_\infty$.

\begin{proposition}

Let $\msa$ be an $Op^*$-algebra on a pre Hilbert space $\ms{D}$ such that $\ms{D}$ is 
complete with respect to $\mc{T}_\msa$, then $\ms{D}_\infty$ is complete with 
respect to $\mc{T}_{\msa_\infty}$.

\begin{proof}

	Let $\{f_\alpha\}_\alpha$ be Cauchy with respect to 
	$\mc{T}_{\msa_\infty}$. Take the $\msa$-amplification consisting 
	of an operator $A \in \msa$ in the n$^{th}$ diagonal entry. We have, by our 
	assumption, that 
	$\norm{A(f_\alpha(n) - f_\beta(n))} = \norm{f_\alpha(n) - f_\beta(n)}_A$ 
	converges to 0. Whence we extract a limit $f(n) = \lim_\alpha f_\alpha(n)$ by
	the completeness of $\ms{D}$. The seminorm $\norm{f_\alpha - f}_X$, 
	where $X$ is any $\msa$-amplification, is simply a finite sum of seminorms 
	on $\ms{D}$ applied to $f_\alpha(j) - f(j)$, $j \le n$ for some 
	$n \in \mathbb{N}$, which converges to zero.
\end{proof}

\end{proposition}

This property is helpful for when it is important that our $Op^*$-algebra is complete, as it allows us to still get a handle on the operators and vectors on this amplified space. 
It also meshes well with how amplifications are defined in the bounded setting.

We wish to apply channels of the original algebras to this amplified space. In
order to use topologies on $\msa_\infty$ in this setting it's convenient to have a
natural embedding of $\msa$ into $\msa_\infty$, motivating why we take infinite matrices
and finitely supported vectors.

\subsection{Topologizing the Channels}

In the previous section we discussed a variety of generalizations of the operator norm
on the operator algebras themselves. We must now discuss how to generalize the completely
bounded norm given a locally convex topology on the algebras.

Let $\pair{\msa_\infty}{\{q_\alpha\}}$ and $\pair{\msb_\infty}{\{q_\beta\}}$
be locally convex amplification algebras, let $\mc{E}:\msa \to \msb$ be a quantum
channel continuous relative to both topologies. Let $\norm{\cdot}_{S,\beta}$
be the family of seminorms defined via 
\[\norm{\mc{E}}_{S, \beta} = \sup_{X \in S}(q_\beta(\mc{E}_\infty(X))),\]
where $S$ is a bounded subset of $\pair{\msa}{\{q_\alpha\}}$. This gives a topology 
on the channels relative to whatever locally convex topology for $\msa$ we choose. 

There is one more step to make a 'completely bounded' topology out of this.

\end{document}
