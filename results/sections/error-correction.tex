\documentclass[../main.tex]{subfiles}

\title{Error Correction}
\author{Jack Wile}
\date{}

\begin{document}

Here we discuss correctability and privacy for quantum channels, now taken
to generality of $Op^*$-algebras as defined in the preliminary section. We
are interested in two kinds of each notion; the exact case and the 
approximate case. The former is merely an algebraic notion, and is little 
different than in the bounded case, save for the generality of the objects.
As discussed in the previous section however, we have a variety of 
generalizations of the operator norm which may or may not agree in the 
unbounded case, which leaves how best to think of correctability 
potentially more ambiguous, certainly compared to the setting which we are
trying to generalize.

It's worth noting still however that even in the bounded case there remains
some ambiguity with what the best notion of correctability and privacy are. First, let's state the definitions we are aiming to extend to unbounded 
algebras.

\begin{definition}

	Let $\mc{E}: \msa \to \msb$ be a quantum channel of \cstar-algebras. A
	\cstar-subalgebra $\ms{N} \subseteq \msb$ is said to be $\textbf{private}$
	for $\mc{E}$ if \[\mc{E}(\msa) \subseteq \ms{N}'.\] Similarly, we saw that 
	$\ms{N}$ is $\varepsilon$-private for $\mc{E}$ if there exists a 
	quantum channel $\mc{F}: \msa \to \msb$, private for $\ms{N}$, such that 
	\[\norm{\mc{E} - \mc{F}}_{\text{cb}} < \varepsilon.\]

\end{definition}

\begin{definition}

	Let $\mc{E}: \msa \to \msb$ be a quantum channels of \cstar-algebras. 
	a \cstar-subalgebra $\ms{N} \subseteq \msb$ is $\textbf{correctable}$
	for $\mc{E}$ if there exists a quantum channel $\mc{R}: \ms{N} \to \ms{A}$ 
	such that \[\mc{E}\mc{R} = \mc{I}_\ms{N}.\] $\ms{N}$ is $\varepsilon$-correctable
	for $\mc{E}$ if instead we have 
	\[\norm{\mc{E}\mc{R} - \mc{I}_\ms{N}}_{\text{cb}} < \varepsilon.\]

\end{definition}

Though we cannot write down the completely bounded norm for an unbounded channel, we can
topologize the channel in a manner which is both more general and captures the 
approximation up to "arbitrary amplification" in a manner similar to the completely 
bounded norm.

However, completely bounded convergence or pertubation is arguably too strong of a 
requirement to describe certain physical pertubations. It is then perhaps desirable to 
develop a correspondence between privacy and correctability in a weaker setting.

We also note that in the above definitions are often restricted to the case where 
$\msa$ is a von Neumann algebra, but the definition makes sense at this level of 
generality and is immediately analogous to our $Op^*$-algebra setting.

\subsection{$\msa$-amplifications}

We aim to work with correctability and privacy in the sense of one of a few possible 
locally convex topologies. It makes sense then to describe these in terms of the 
seminorms of whatever topology we choose. What isn't immediately clear is how to capture
the 'amplified' nature of the completely bounded topology, and so first a few things are
in order.

\begin{definition}

Let $\msa$ be an $Op^*$-algebra. We denote by $\msa_n = \msa \otimes M_n$ 
the space of n by n $\msa$-matrices, which we may verbally refer to as the
$\msa$-amplifications of degree n.

\end{definition}

$\msa_n$ is an $Op^*$-algebra when acting on the direct sum of n copies of 
$\msd$, which we write as $\msd_n$. In this setting we have stability of the 
completeness of the domain.

\begin{proposition}

Let $\msa$ be an $Op^*$-algebra on a pre Hilbert space $\ms{D}$ such that $\ms{D}$ is 
complete with respect to $\mc{T}_\msa$, then $\ms{D}_n$ is complete with 
respect to $\mc{T}_{\msa_n}$.

\begin{proof}

	Let $\{f_\alpha\}_\alpha$ be Cauchy with respect to 
	$\mc{T}_{\msa_n}$. Take the $\msa$-amplification consisting 
	of an operator $A \in \msa$ in the k$^{th}$ diagonal entry. We have, by our 
	assumption, that 
	$\norm{A(f_\alpha(k) - f_\beta(k))} = \norm{f_\alpha(k) - f_\beta(k)}_A$ 
	converges to 0. Whence we extract a limit $f(k) = \lim_\alpha f_\alpha(k)$ by
	the completeness of $\ms{D}$. The seminorm $\norm{f_\alpha - f}_X$, 
	where $X$ is any $\msa$-amplification, is simply a finite sum of seminorms 
	on $\ms{D}$ applied to $f_\alpha(k) - f(k)$, which converges to 0.
\end{proof}

\end{proposition}

There is of course a natural embedding of $\msa$ into $\msa_n$ via the map 
$A \to A \otimes 1_n$. It would be convenient if any of the locally convex 
topologies one can put on an $Op^*$-algebra would be stable under taking the 
subspace topology with respect to this embedding. Thus we will assume that this 
embedding is a homeomorphism with respect to the topology taken for our algebras.

\subsection{Topologizing the Channels}

In the previous section we discussed a variety of generalizations of the operator norm
on the operator algebras themselves. We must now discuss how to generalize the completely
bounded norm given a locally convex topology on the algebras.

Let $\pair{\msa_n}{\{q_\alpha\}}$ and $\pair{\msb_n}{\{q_\beta\}}$
be locally convex amplification algebras, let $\mc{E}:\msa \to \msb$ be a quantum
channel continuous relative to both topologies.

\end{document}
